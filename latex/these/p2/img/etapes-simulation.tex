\documentclass{standalone}
\usepackage[utf8]{inputenc}
\usepackage[T1]{fontenc}
\usepackage[french]{babel}

\usepackage{mathptmx}
\usepackage{tikz}

\usetikzlibrary{arrows}
\usetikzlibrary{shapes}
\usetikzlibrary{positioning}
\usetikzlibrary{fit}

\begin{document}
\begin{tikzpicture}
	\tikzstyle{etape}=[rounded corners,rectangle split,rectangle split parts=2,draw,align=center,text width=4cm]
	\tikzstyle{arrow}=[>=stealth,->,thick,rounded corners]
	
	\node [thick,gray!100] (timestep) {Pas de temps};

	\node [below right=of timestep.north west,etape] (force) {
			\textbf{Force} 
			\nodepart{second} 
			Calcul des forces nodales
	};

  	\node [below=of force,etape] (mobilite) {
			\textbf{Mobilit\'e} 
			\nodepart{second} 
			Application des lois de Mobilit\'e
	};

	\node [below=of mobilite,etape] (deplacement) {
			\textbf{D\'eplacement} 
			\nodepart{second} 
			Nouvelle position des noeuds.
	};

	\node [below=of deplacement,etape] (local) {
			\textbf{Op\'erations locales} 
			\nodepart{second} 
			Collision \\ Splitnode \\ Remesh			
	};

	\draw [arrow] (force) -> node [right] {$F_t$} (mobilite) ;
	\draw [arrow]  (mobilite) -> node [right] {$V_t$} (deplacement) ;
	\draw [arrow]  (deplacement) -> node [right] {$P^\prime$} (local) ;
	\draw [arrow]  (force.west)  --  +(-1,0)  |-   node [right,pos=0.2] {$F_t$} (local.west) ;

	\node [right=of force,fill,circle] (begin) {};

	\draw [arrow]  (local.south) -- node [right] {$P_{t+1}$} +(0,-1) -- +(5,-1) |- node [right,pos=0.25] {$t++$} (begin);

	\draw[arrow] (begin) -- node [above] {$P_t$} (force);
	\draw[arrow] (begin) |- node [above,pos=0.75] {$P_t$} (deplacement);
	\draw[arrow] (begin) |- node [above,pos=0.75] {$P_t$} (local);

	\node [rounded corners,thick,draw=gray!50, fit={(timestep) (force) (mobilite) (deplacement) (local) (begin)}] {};

\end{tikzpicture}
\end{document}