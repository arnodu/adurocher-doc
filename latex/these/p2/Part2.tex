\documentclass[11pt,class=article,float=false,crop=false]{standalone}
\usepackage{Part2_packages}

\title{ Partie 2 }
\author{Arnaud Durocher}

\begin{document}
	
\onlyifstandalone{\maketitle}
\onlyifstandalone{\tableofcontents}
\onlyifstandalone{\listoftodos}

\part{Structure de données pour la Dynamique des Dislocations.}

La manière de stocker les données de la simulation est d'une importance cruciale à la fois pour la validité de la simulation, la flexibilité des algorithmes, et la performance. Le travail effectué ici vise à proposer une structure de données adaptée à la dynamique des dislocations qui permet d'implémenter facilement des algorithmes l'utilisant tout en permettant une flexibilité dans son implémentation afin de pouvoir expérimenter sur la performance.

La structure de données pour la dynamique des dislocations doit répondre à certaines contraintes qui seront présentées en 1ere section. Nous définirons ensuite un type abstrait de données qui permettra de répondre à ces contraintes. Nous verrons ensuite comment implémenter cette interface, notamment dans un environnement parallèle et distribué. Enfin nous étudierons la performance de cette structure de données.

\import{./}{1-Definitions}
\import{./}{2-TAD}
\import{./}{3-Implementation}
\import{./}{4-Perf}


\bibliographystyledata{plain}
\bibliographydata{Part2_biblio}

\onlyifstandalone{
\begin{appendices}
    \import{./}{Part2_appendix}
\end{appendices}
}

\end{document}