\documentclass{standalone}
\usepackage[utf8]{inputenc}
\usepackage[T1]{fontenc}
\usepackage[french]{babel}

\usepackage{mathptmx}
\usepackage{tikz}

\usetikzlibrary{arrows}
\usetikzlibrary{shapes}
\usetikzlibrary{positioning}

\begin{document}
\begin{tikzpicture}
	\tikzstyle{etape}=[rounded corners,rectangle split,rectangle split parts=2,draw,align=center,text width=5cm]
	\tikzstyle{arrow}=[>=stealth,->,thick,rounded corners]
	\tikzstyle{decision} = [diamond, draw, text badly centered, inner sep=3pt]


	\node [rounded corners,draw,align=center,text width=2cm,thick] (begin) { \textbf{Début} };

	\node [below=0.5cm of begin, decision,aspect=2] (s0) {
			\textbf{Liste des collisions} 
	};

	\node [below=1cm of s0,etape] (s1) {
			\textbf{D\'eplacement} 
			\nodepart{second} 
			Tous les noeuds sont déplacés au temps $t_c$.
	};

  	\node [below=0.5cm of s1,etape] (s2) {
			\textbf{Operation topologique} 
			\nodepart{second} 
			Fusion des objets $o_1$ et $o_2$.
	};

	\node [below=0.5cm of s2,etape] (s3) {
			\textbf{Recalcul} 
			\nodepart{second} 
			Calcul des collisions \\ générées par les modifications.
	};

	\node [left=1cm of s0, rounded corners,draw,align=center,text width=1cm,thick] (end) { \textbf{Fin} };

	\draw [arrow] (begin) -> (s0) ;
	\draw [arrow] (s0) -> node [fill=white] { \footnotesize $1^{re}$ collision: $\{t_c,o_1,o_2\}$} (s1) ;
	\draw [arrow] (s0) -> node [above] { \footnotesize Vide} (end) ;
	\draw [arrow] (s1) -> (s2) ;
	\draw [arrow] (s2) -> (s3) ;

	\draw [arrow] (s3.south) |- +(4cm,-0.5cm) |-  node [fill=white,pos=0.25,align=center] {\footnotesize Collision \\ \footnotesize suivante} (s0.east);


\end{tikzpicture}
\end{document}